\documentclass[a4paper,twoside]{article}
\usepackage[T1]{fontenc}
\usepackage[bahasa]{babel}
\usepackage{graphicx}
\usepackage{graphics}
\usepackage{float}
\usepackage[cm]{fullpage}
\pagestyle{myheadings}
\usepackage{etoolbox}
\usepackage{setspace} 
\usepackage{lipsum} 
\setlength{\headsep}{30pt}
\usepackage[inner=2cm,outer=2.5cm,top=2.5cm,bottom=2cm]{geometry} %margin
% \pagestyle{empty}

\makeatletter
\renewcommand{\@maketitle} {\begin{center} {\LARGE \textbf{ \textsc{\@title}} \par} \bigskip {\large \textbf{\textsc{\@author}} }\end{center} }
\renewcommand{\thispagestyle}[1]{}
\markright{\textbf{\textsc{AIF234001/AIF234002 \textemdash Rencana Kerja Tugas Akhir \textemdash Sem. Ganjil 2023/2024}}}

\newcommand{\HRule}{\rule{\linewidth}{0.4mm}}
\renewcommand{\baselinestretch}{1}
\setlength{\parindent}{0 pt}
\setlength{\parskip}{6 pt}

\onehalfspacing

\begin{document}
	
	\title{\@judultopik}
	\author{\nama \textendash \@npm} 
	
	%tulis nama dan NPM anda di sini:
	\newcommand{\nama}{Nicholas Khrisna Sandyawan}
	\newcommand{\@npm}{6181801060}
	\newcommand{\@judultopik}{Evaluasi PHP Standards Recommendations pada proyek SharIF Judge} % Judul/topik anda
	\newcommand{\jumpemb}{1} % Jumlah pembimbing, 1 atau 2
	\newcommand{\tanggal}{27/09/2023}
	
	% Dokumen hasil template ini harus dicetak bolak-balik !!!!
	
	\maketitle
	
	\pagenumbering{arabic}
	
	\section{Deskripsi}
	SharIF Judge merupakan perangkat lunak berbasis web yang dapat digunakan untuk menilai kode program dalam bahasa C, C++, Java, dan Python. SharIF Judge yang dibahas pada tugas akhir ini adalah \textit{fork} dari Sharif Judge yang dibuat oleh Mohammad Javad Naderi. Versi \textit{fork} ini sudah dikembangkan sesuai kebutuhan jurusan Teknik Informatika UNPAR dalam proses penilaian di beberapa mata kuliah.
	
	Dalam pengembangan perangkat lunak, penerapan aturan atau standar tertentu akan memudahkan orang lain dalam memahami kode program yang dibuat terutama jika dikerjakan oleh lebih dari satu orang. PHP Standards Recommendations (PSR) adalah kumpulan rekomendasi standar penulisan PHP yang dibuat oleh PHP Framework Interop Group. Pada saat dokumen ini dibuat, terdapat 14 bab yang sudah diterima (Accepted) untuk digunakan, 4 bab masih didiskusikan (Draft), 3 bab ditinggalkan (Abandoned), dan 2 bab sudah usang (Deprecated). Hanya bab-bab  berstatus ``Accepted'' yang akan digunakan sebagai acuan untuk rekomendasi. Bab-bab tersebut antara lain: 
	\begin{itemize}
		\item PSR-01: Basic Coding Standard
		\item PSR-03: Logger Interface
		\item PSR-04: Autoloading Standard
		\item PSR-06: Caching Interface
		\item PSR-07: HTTP Message Interface
		\item PSR-11: Container Interface
		\item PSR-12: Extended Coding Style Guide
		\item PSR-13: Hypermedia Links
		\item PSR-14: Event Dispatcher
		\item PSR-15: HTTP Handlers
		\item PSR-16: Simple Cache
		\item PSR-17: HTTP Factories
		\item PSR-18: HTTP Client
		\item PSR-20: Clock
	\end{itemize}
	
	Pada tugas akhir ini, \textit{file}-\textit{file} PHP pada SharIF Judge akan dievaluasi tingkat kepatuhannya terhadap rekomendasi standar PSR. Selanjutnya akan diimplementasikan rekomendasi dari PSR berdasarkan hasil evaluasi tersebut. Walaupun demikian, masih akan ditentukan strategi untuk melakukannya, misalnya penggunaan \textit{tool} tertentu untuk membantu pemeriksaan, bab-bab apa saja yang relevan untuk dievaluasi sesuai yang digunakan pada SharIF Judge, dan seberapa banyak bab yang harus diperiksa secara manual. Salah satu \textit{tool} yang digunakan adalah PHP Coding Standards Fixer (PHP CS Fixer). PHP CS Fixer membantu proses pemeriksaan sesuai salah satu rekomendasi standar, yaitu PSR-12: Extended Coding Style Guide. Sejauh ini, belum ada \textit{tool} yang dapat membantu evaluasi terhadap bab-bab rekomendasi PSR lainnya.
	
	\section{Rumusan Masalah}
	\begin{enumerate}
		\item Bagaimana tingkat kepatuhan kode PHP pada SharIF Judge terhadap PSR?  
		\item Rekomendasi perbaikan apa yang dapat diberikan pada kode PHP SharIF Judge untuk meningkatkan jumlah rekomendasi PSR yang terpenuhi?
	\end{enumerate}
	
	\section{Tujuan}
	\begin{enumerate}
		\item Mengukur tingkat kepatuhan kode PHP pada SharIF Judge terhadap PSR.
		\item Memberikan rekomendasi perbaikan sesuai PSR pada kode PHP SharIF Judge agar meningkatkan jumlah rekomendasi yang terpenuhi. 
	\end{enumerate}
	
	\section{Deskripsi Perangkat Lunak}
	Hasil akhir perangkat lunak yang akan dibuat memiliki fitur yang sama persis dengan SharIf Judge yang sudah ada. Perbedaannya terdapat pada struktur penulisan PHP yang sudah dievaluasi dan diperbaiki sesuai rekomendasi  PSR. 
	
	\section{Detail Pengerjaan Tugas Akhir}
	Bagian-bagian pekerjaan tugas akhir ini adalah sebagai berikut :
	\begin{enumerate}
		\item Mempelajari SharIF Judge 
		\item Melakukan studi literatur mengenai PSR dan PHP CS Fixer
		\item Mengevaluasi tingkat kepatuhan SharIF Judge terhadap PSR
		\item Mengimplementasikan rekomendasi PSR sesuai hasil evaluasi
		\item Menguji SharIF Judge yang sudah dievaluasi dan diperbaiki
		\item Menulis dokumen tugas akhir
	\end{enumerate}
	
	\section{Rencana Kerja}
	Rincian capaian yang direncanakan di Tugas Akhir 1 adalah sebagai berikut:
	\begin{enumerate}
		\item Mempelajari SharIF Judge
		\item Melakukan studi literatur mengenai PSR dan PHP CS Fixer
		\item Menentukan strategi untuk mengevaluasi keseluruhan PHP SharIF Judge
		\item Mengevaluasi sebagian PHP pada SharIF Judge, khususnya yang dapat dilakukan dengan bantuan PHP CS Fixer
		\item Menulis sebagian dokumen tugas akhir yaitu bab 1, 2, dan 3
	\end{enumerate}
	
	Sedangkan yang akan diselesaikan di Tugas Akhir 2 adalah sebagai berikut:
	\begin{enumerate}
		\item Mengevaluasi sebagian PHP pada SharIF Judge, khususnya yang harus dilakukan secara manual sesuai strategi yang sudah dibuat
		\item Membuat rekomendasi perbaikan kode PHP sesuai hasil evaluasi
		\item Melakukan pengujian terhadap SharIF Judge yang sudah dievaluasi dan diperbaiki
		\item Menulis dan melengkapi dokumen tugas akhir untuk bab 4 dan 5
		
	\end{enumerate}
	
	\vspace{1cm}
	\centering Bandung, \tanggal\\
	\vspace{2cm} \nama \\ 
	\vspace{1cm}
	
	Menyetujui, \\
	\ifdefstring{\jumpemb}{2}{
		\vspace{1.5cm}
		\begin{centering} Menyetujui,\\ \end{centering} \vspace{0.75cm}
		\begin{minipage}[b]{0.45\linewidth}
			% \centering Bandung, \makebox[0.5cm]{\hrulefill}/\makebox[0.5cm]{\hrulefill}/2013 \\
			\vspace{2cm} Nama: \makebox[3cm]{\hrulefill}\\ Pembimbing Utama
		\end{minipage} \hspace{0.5cm}
		\begin{minipage}[b]{0.45\linewidth}
			% \centering Bandung, \makebox[0.5cm]{\hrulefill}/\makebox[0.5cm]{\hrulefill}/2013\\
			\vspace{2cm} Nama: \makebox[3cm]{\hrulefill}\\ Pembimbing Pendamping
		\end{minipage}
		\vspace{0.5cm}
	}{
		% \centering Bandung, \makebox[0.5cm]{\hrulefill}/\makebox[0.5cm]{\hrulefill}/2013\\
		\vspace{2cm} Nama: \makebox[3cm]{\hrulefill}\\ Pembimbing Tunggal
	}
\end{document}
