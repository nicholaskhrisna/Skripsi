\documentclass[a4paper,twoside]{article}
\usepackage[T1]{fontenc}
\usepackage[bahasa]{babel}
\usepackage{graphicx}
\usepackage{graphics}
\usepackage{float}
\usepackage[cm]{fullpage}
\pagestyle{myheadings}
\usepackage{etoolbox}
\usepackage{setspace} 
\usepackage{lipsum} 
\setlength{\headsep}{30pt}
\usepackage[inner=2cm,outer=2.5cm,top=2.5cm,bottom=2cm]{geometry} %margin
% \pagestyle{empty}

\makeatletter
\renewcommand{\@maketitle} {\begin{center} {\LARGE \textbf{ \textsc{\@title}} \par} \bigskip {\large \textbf{\textsc{\@author}} }\end{center} }
\renewcommand{\thispagestyle}[1]{}
\markright{\textbf{\textsc{AIF401/AIF402 \textemdash Rencana Kerja Skripsi \textemdash Sem. Genap 2022/2023}}}

\newcommand{\HRule}{\rule{\linewidth}{0.4mm}}
\renewcommand{\baselinestretch}{1}
\setlength{\parindent}{0 pt}
\setlength{\parskip}{6 pt}

\onehalfspacing
 
\begin{document}

\title{\@judultopik}
\author{\nama \textendash \@npm} 

%tulis nama dan NPM anda di sini:
\newcommand{\nama}{Nicholas Khrisna Sandyawan}
\newcommand{\@npm}{6181801060}
\newcommand{\@judultopik}{Evaluasi PHP Standards Recommendations pada proyek SharIF Judge} % Judul/topik anda
\newcommand{\jumpemb}{1} % Jumlah pembimbing, 1 atau 2
\newcommand{\tanggal}{01/01/1900}

% Dokumen hasil template ini harus dicetak bolak-balik !!!!

\maketitle

\pagenumbering{arabic}

\section{Deskripsi}
SharIF Judge merupakan perangkat lunak berbasis web yang dapat digunakan oleh mahasiswa Teknik Informatika UNPAR. SharIF Judge dibuat oleh Mohammad Javad Naderi. Mahasiswa dapat mengunggah kode program dan akan dinilai kebenarannya secara otomatis oleh SharIF Judge.

PHP Standards Recommendations (PSR) adalah kumpulan standar penulisan PHP yang dibuat oleh PHP Framework Interop Group. Pada saat skripsi ini dibuat, terdapat 14 standar yang sudah diterima dan digunakan.

Pada skripsi ini, keseluruhan PHP pada SharIF Judge akan dilihat dan dievaluasi seberapa jauh standar PSR yang sudah dipenuhi. Alat yang digunakan adalah PHP linter yang dapat membantu proses pemeriksaan sesuai salah satu standar, yaitu Extended Coding Style Guide atau aturan penulisan PHP yang sudah diperbarui. Sementara itu standar lainnya diperiksa secara manual. 

\section{Rumusan Masalah}
Rumusan masalah yang akan dibahas pada skripsi ini sebagai berikut:
\begin{itemize}
	\item Apakah SharIF Judge sudah memenuhi PSR?
	\item Bagaimana mengevaluasi SharIF Judge agar sesuai dengan PSR?  
\end{itemize}

\section{Tujuan}
Tujuan yang ingin dicapai dalam penulisan skripsi ini sebagai berikut:
\begin{itemize}
	\item Mengetahui seberapa jauh PSR yang sudah terpenuhi pada SharIF Judge.
	\item Membuat evaluasi SharIF Judge agar sesuai dengan PSR.
\end{itemize}

\section{Deskripsi Perangkat Lunak}
Perangkat lunak akhir yang akan dibuat memiliki fitur yang sama persis dengan SharIf Judge yang sudah ada. Perbedaannya terdapat pada struktur php yang sudah dievaluasi sehingga memenuhi PSR.

\section{Detail Pengerjaan Skripsi}
Bagian-bagian pekerjaan skripsi ini adalah sebagai berikut :
	\begin{enumerate}
		\item Memperlajari SharIF Judge saat ini
		\item Melakukan studi literatur mengenai PHP linter dan PSR
		\item Mengevaluasi PHP dari SharIF Judge sesuai dengan PSR
		\item Menguji SharIF Judge yang sudah dievaluasi
		\item Menulis dokumen skripsi
	\end{enumerate}

\section{Rencana Kerja}
Rincian capaian yang direncanakan di Skripsi 1 adalah sebagai berikut:
\begin{enumerate}
\item Mempelajari SharIF Judge saat ini
\item Melakukan studi literatur mengenai PHP linter dan PSR
\item Mengevaluasi sebagian kecil PHP pada SharIF Judge dengan PHP linter
\item Menulis sebagian dokumen skripsi yaitu bab 1, 2, dan 3
\end{enumerate}

Sedangkan yang akan diselesaikan di Skripsi 2 adalah sebagai berikut:
\begin{enumerate}
\item Mengevaluasi sebagian besar PHP pada SharIF Judge secara manual
\item Melakukan pengujian terhadap SharIF Judge yang sudah dievaluasi
\item Menulis dan melengkapi dokumen skripsi untuk bab 4, 5, dan 6

\end{enumerate}

\vspace{1cm}
\centering Bandung, \tanggal\\
\vspace{2cm} \nama \\ 
\vspace{1cm}

Menyetujui, \\
\ifdefstring{\jumpemb}{2}{
\vspace{1.5cm}
\begin{centering} Menyetujui,\\ \end{centering} \vspace{0.75cm}
\begin{minipage}[b]{0.45\linewidth}
% \centering Bandung, \makebox[0.5cm]{\hrulefill}/\makebox[0.5cm]{\hrulefill}/2013 \\
\vspace{2cm} Nama: \makebox[3cm]{\hrulefill}\\ Pembimbing Utama
\end{minipage} \hspace{0.5cm}
\begin{minipage}[b]{0.45\linewidth}
% \centering Bandung, \makebox[0.5cm]{\hrulefill}/\makebox[0.5cm]{\hrulefill}/2013\\
\vspace{2cm} Nama: \makebox[3cm]{\hrulefill}\\ Pembimbing Pendamping
\end{minipage}
\vspace{0.5cm}
}{
% \centering Bandung, \makebox[0.5cm]{\hrulefill}/\makebox[0.5cm]{\hrulefill}/2013\\
\vspace{2cm} Nama: \makebox[3cm]{\hrulefill}\\ Pembimbing Tunggal
}
\end{document}

