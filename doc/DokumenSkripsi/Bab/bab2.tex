%versi 3 (22-07-2020)
\chapter{Landasan Teori}
\label{chap:teori}

\section{SharIF Judge}
\label{sec:sharifjudge} 
 
SharIF Judge (dengan huruf kapital "IF") merupakan perangkat lunak berbasis web yang digunakan untuk menilai kode program dalam bahasa C, C++, Java, dan Python. SharIF Judge yang dibahas dalam dokumen ini adalah versi fork dari Sharif Judge (dengan huruf kecil "if") yang dibuat oleh Mohammad Javad Naderi. Versi fork ini dikembangkan sesuai dengan kebutuhan jurusan Teknik Informatika UNPAR dalam proses penilaian di beberapa mata kuliah. SharIF judge dibuat dengan PHP pada framework CodeIgniter dan BASH untuk backend.

\subsection{Fitur}
Berikut adalah fitur-fitur dari SharIF Judge.
\begin{enumerate}
	\item  Terdapat beberapa role pengguna, antara lain admin, head instructor, instructor, dan student.
	\item  Sandboxing (belum tersedia untuk Python)
	\item Deteksi kecurangan (mendeteksi kemiripan kode) menggunakan Moss
	\item Pengaturan untuk keterlambatan pengumpulan
	\item Antrian pengiriman
	\item Mengunduh hasil dalam bentuk file excel
	\item Mengunduh kode yang dikirim dalam bentuk file zip
	\item Metode "Output Comporison" dan "Tester Code" untuk memeriksa kebenaran dari hasil output
	\item Menambahkan beberapa pengguna sekaligus
	\item Deskripsi masalah (PDF/Markdown/HTML)
	\item Penilaian ulang (rejudge)
	\item Papan Nilai (Scoreboard) dan Notifikasi (Notifications)
	
\end{enumerate}

\subsection{Instalasi}
Untuk menjalankan SharIF Judge, dibutuhkan sebuah server Linux dengan syarat sebagai berikut:
\begin{itemize}
	\item Webserver menjalankan PHP versi 5.3 atau lebih baru
	\item Pengguna dapat menjalankan PHP melalui command line.
	\item Menggunakan database MySql atau PostgreSql.
	\item PHP harus diberikan akses untuk untuk menjalankan perintah menggunakan fungsi shell_exec.
	\item Tools yang digunakan untuk melakukan kompilasi dan menjalankan kode yang dikumpulkan.
	\item Perl lebih baik diinstal untuk alasan ketepatan waktu, batas memori dan memaksimalkan batas ukuran pada output kode yang dikirim.
\end{itemize}



%\dtext{11-12} 

%\section{\LaTeX}
\section{PHP Standards Recommendations}
\label{sec:psr}

PHP Standards Recommendations (PSR) adalah kumpulan rekomendasi yang dibuat oleh PHP Framework Interop Group (PHP-FIG) untuk membantu para pengembang PHP dalam menciptakan kode yang lebih mudah dibaca, dipahami, dan dipelihara. Setiap PSR memiliki status, antara lain Accepted, Draft, Abandoned, dan Deprecated. 
Saat dokumen ini dibuat, terdapat 23 bab PSR dengan rincian sebagai berikut.
\begin{itemize}
	\item{Accepted}
	\begin{enumerate}
		\item (1) Basic Coding Standard
		\item (3) Logger Interface
		\item (4) Autoloading Standard
		\item (6) Caching Interface
		\item (7) HTTP Message Interface
		\item (11) Container Interface
		\item (12) Extended Coding Style Guide
		\item (13) Hypermedia Links
		\item (14) Event Dispatcher
		\item (15) HTTP Handlers
		\item (16) Simple Cache
		\item (17) HTTP Factories
		\item (18) HTTP Client
		\item (20) Clock
	\end{enumerate}

	\item{Draft}
	\begin{enumerate}
		\item (5) PHPDoc Standard
		\item (19) PHPDoc Tags
		\item (21) Internationalization
		\item (22) Application Tracing
	\end{enumerate}
	
	\item{Abandoned}
	\begin{enumerate}
		\item (8) Huggable Interface
		\item (9) Security Advisories
		\item (10) Security Reporting Process
	\end{enumerate}
	
	\item{Deprecated}
	\begin{enumerate}
		\item (0) Autoloading Standard
		\item (2) Coding Style Guide
	\end{enumerate}
	
	
\end{itemize}

%\dtext{13-14}


\section{PHP Linter}
\label{sec:phplinter}
 
Lint awalnya merujuk pada tool yang digunakan untuk menganalisis suatu kode program dengan tujuan menemukan kesalahan pada bahasa C. Kemudian istilah ini menjadi sebutan untuk mendeskripsikan hal-hal yang berkaitan dengan pengecekan kode program. PHP linter adalah tool yang digunakan untuk menganalisis kode PHP sesuai dengan standar tertentu. 
PHP linter yang digunakan adalah yang dibuat oleh Brueggern. Linter ini berdasar pada standar PSR ke-2 dan ke-12, yaitu Coding Style Guide yang sudah usang (deprecated) dan Extended Coding Style Guide sebagai penggantinya (accepted). 

%\dtext{15-16}

\subsection{Syarat Instalasi}
Sebelum menginstal linter, perlu dilakukan penginstalan Composer terlebih dahulu. Composer adalah alat untuk mengelola dependency pada PHP. 

\subsection{Instalasi}
Berikut adalah langkah instalasi PHP linter.
\begin{enumerate}
	\item Pada bagian root project, buka file composer.json.
	\item Pada bagian "Repositories", tambahkan kode berikut.
	\begin{lstlisting}[language=php, caption=kode kode, label=kode:aneh]
		{
			"repositories": [
			{
				"type": "vcs",
				"url": "git@github.com:brueggern/php-linter.git"
			}
			]
		}
		
	\end{lstlisting}
	
	\item Install composer package.
	\verb|composer require brueggern/php-linter|  
	
	\item Tambahkan script berikut untuk menjalankan linting/fixing. "app" dapat diganti dengan nama file atau folder yang akan di-lint. 
	\begin{lstlisting}[language=php, caption=kode kode, label=kode:aneh]
		{
			"scripts": {
				"lint": "php-linter app",
				"lint:fix": "php-linter --fix app"
			}
		}
		
	\end{lstlisting}
\end{enumerate}


\subsection{Penggunaan}
Untuk menjalankan linter perintah yang digunakan adalah:
	\verb|composer run lint|    

Untuk memperbaiki error secara otomatis, perintah yang digunakan adalah:

	\verb|composer run lint:fix|    


\subsection{Tabel}  
Berikut adalah contoh pembuatan tabel. 
Penempatan tabel dan gambar secara umum diatur secara otomatis oleh \LaTeX{}, perhatikan contoh di file bab2.tex untuk melihat bagaimana cara memaksa tabel ditempatkan sesuai keinginan kita.

Perhatikan bawa berbeda dengan penempatan judul gambar gambar, keterangan tabel harus diletakkan di atas tabel!!
Lihat Tabel~\ref{tab:contoh1} berikut ini:

\begin{table}[H] %atau h saja untuk "kira kira di sini"
	\centering 
	\caption{Tabel contoh}
	\label{tab:contoh1}
	\begin{tabular}{cccc}
		\toprule
		& $v_{start}$ & $\mathcal{S}_{1}$ & $v_{end}$\\

		\midrule
		$\tau_{1}$ & 1 & 12& 20\\
		$\tau_{2}$ & 1 &  & 20\\
		$\tau_{3}$ & 1 & 9 & 20\\
		$\tau_{4}$ & 1 &  & 20\\

		\bottomrule
		
	\end{tabular} 
\end{table}
Tabel~\ref{tab:cthwarna1} dan Tabel~\ref{tab:cthwarna2} berikut ini adalah tabel dengan sel yang berwarna dan ada dua tabel yang bersebelahan. 
\begin{table}[H]
	\begin{minipage}[c]{0.49\linewidth}
		\centering
		\caption{Tabel bewarna(1)}
		\label{tab:cthwarna1}
		\begin{tabular}{ccccc}
			\toprule
			 & $v_{start}$ & $\mathcal{S}_{2}$ & $\mathcal{S}_{1}$ & $v_{end}$\\
			
			\midrule
			$\tau_{1}$ & 1 & 5 \cellcolor{green}& 12& 20\\
			$\tau_{2}$ & 1 & 8 \cellcolor{green}& & 20\\
			$\tau_{3}$ & 1 & 2/8/17 \cellcolor{green}& 9 & 20\\
			$\tau_{4}$ & 1 & \cellcolor{red}& & 20\\
			
			\bottomrule

		\end{tabular}
	\end{minipage}
	\begin{minipage}[c]{0.49\linewidth}
		
		\centering 
		\caption{Tabel bewarna(2)}
		\label{tab:cthwarna2}
		\begin{tabular}{ccccc}
			\toprule
			 & $v_{start}$ & $\mathcal{S}_{1}$ & $\mathcal{S}_{2}$ & $v_{end}$\\
			
			\midrule
			$\tau_{1}$ & 1 & 12& 5 \cellcolor{red} &20\\
			$\tau_{2}$ & 1 &  &  8 \cellcolor{green} &20\\
			$\tau_{3}$ & 1 & 9 & 2/8/17 \cellcolor{green} &20\\
			$\tau_{4}$ & 1 &   & \cellcolor{red} &20\\
			
			\bottomrule
		
		\end{tabular}
	\end{minipage}
\end{table}

 
\subsection{Kutipan}
\label{subs:kutipan} 
Berikut contoh kutipan dari berbagai sumber, untuk keterangan lebih lengkap, silahkan membaca file referensi.bib yang disediakan juga di template ini.
Contoh kutipan:
\begin{itemize}
	\item Buku:~\cite{berg:08:compgeom} 
	\item Bab dalam buku:~\cite{kreveld:04:GIS}
	\item Artikel dari Jurnal:~\cite{buchin:13:median}
	\item Artikel dari prosiding seminar/konferensi:~\cite{kreveld:11:median}
	\item Skripsi/Thesis/Disertasi:~\cite{lionov:02:animasi}~\cite{wiratma:10:following}~\cite{wiratma:22:later}
	\item Technical/Scientific Report:~\cite{kreveld:07:watertight}
	\item RFC (Request For Comments):~\cite{RFC1654}
	\item Technical Documentation/Technical Manual:~\cite{Z.500}~\cite{unicode:16:stdv9}~\cite{google:16:and7}
	\item Paten:~\cite{webb:12:comm}
	\item Tidak dipublikasikan:~\cite{wiratma:09:median}~\cite{lionov:11:cpoly}
	\item Laman web:~\cite{erickson:03:cgmodel}  
	\item Lain-lain:~\cite{agung:12:tango}
\end{itemize}    
  
\subsection{Gambar}

Pada hampir semua editor, penempatan gambar di dalam dokumen \LaTeX{} tidak dapat dilakukan melalui proses {\it drag and drop}.
Perhatikan contoh pada file bab2.tex untuk melihat bagaimana cara menempatkan gambar.
Beberapa hal yang harus diperhatikan pada saat menempatkan gambar:
\begin{itemize}
	\item Setiap gambar {\bf harus} diacu di dalam teks (gunakan {\it field} {\sc label})
	\item {\it Field} {\sc caption} digunakan untuk teks pengantar pada gambar. Terdapat dua bagian yaitu yang ada di antara tanda $[$ dan $]$ dan yang ada di antara tanda $\{$ dan $\}$. Yang pertama akan muncul di Daftar Gambar, sedangkan yang kedua akan muncul di teks pengantar gambar. Untuk skripsi ini, samakan isi keduanya.
	\item Jenis file yang dapat digunakan sebagai gambar cukup banyak, tetapi yang paling populer adalah tipe {\sc png} (lihat Gambar~\ref{fig:ularpng}), tipe {\sc jpg} (Gambar~\ref{fig:ularjpg}) dan tipe {\sc pdf} (Gambar~\ref{fig:ularpdf})
	\item Besarnya gambar dapat diatur dengan {\it field} {\sc scale}.
	\item Penempatan gambar diatur menggunakan {\it placement specifier} (di antara tanda  $[$ dan $]$ setelah deklarasi gambar.
	Yang umum digunakan adalah {\bf H} untuk menempatkan gambar {\bf sesuai} penempatannya di file .tex atau  {\bf h} yang berarti "kira-kira" di sini. \\
	Jika tidak menggunakan {\it placement specifier}, \LaTeX{} akan menempatkan gambar secara otomatis untuk menghindari bagian kosong pada dokumen anda.
	Walaupun cara ini sangat mudah, hindarkan terjadinya penempatan dua gambar secara berurutan. 	
	\begin{itemize}
		\item Gambar~\ref{fig:ularpng} ditempatkan di bagian atas halaman, walaupun penempatannya dilakukan setelah penulisan 3 paragraf setelah penjelasan ini.
		\item Gambar~\ref{fig:ularjpg} dengan skala 0.5 ditempatkan di antara dua buah paragraf. Perhatikan penulisannya di dalam file bab2.tex!
		\item Gambar~\ref{fig:ularpdf} ditempatkan menggunakan {\it specifier} {\bf h}.
	\end{itemize}
\end{itemize}
 
%\dtext{17-18}
\begin{figure} 
	\centering  
	\includegraphics[scale=1]{ular-png}  
	\caption[Gambar {\it Serpentes} dalam format png]{Gambar {\it Serpentes} dalam format png} 
	\label{fig:ularpng} 
\end{figure} 

%\dtext{19-20}
\begin{figure}[H]
	\centering  
	\includegraphics[scale=0.5]{ular-jpg}  
	\caption[Ular kecil]{Ular kecil} 
	\label{fig:ularjpg} 
\end{figure} 
%\dtext{21-22}

\begin{figure}[ht] 
	\centering  
	\includegraphics[scale=1]{ular-pdf}  
	\caption[ {\it Serpentes} betina]{ {\it Serpentes} jantan} 
	\label{fig:ularpdf} 
\end{figure} 
 
\subsection{Kode Program}

Kode program dalam bahasa tertentu seringkali harus ditulis di dalam bab, bukan hanya dilampirkan di bagian Lampiran. 
Kode~\ref{kode:aneh} menampilkan penggunaan karakter-karakter yang umum digunakan dalam sebuah program yang ditulis dengan bahasa C.


\begin{lstlisting}[language=Java, caption=Kode untuk menampilkan karakter-karakter aneh, label=kode:aneh]
// This does not make algorithmic sense, 
// but it shows off significant programming characters.

#include<stdio.h>

void myFunction( int input, float* output ) {
	switch ( array[i] ) {
		case 1: // This is silly code
			if ( a >= 0 || b <= 3 && c != x )
				*output += 0.005 + 20050;
			char = 'g';
			b = 2^n + ~right_size - leftSize * MAX_SIZE;
			c = (--aaa + &daa) / (bbb++ - ccc % 2 );
			strcpy(a,"hello $@?"); 
	}
	count = ~mask | 0x00FF00AA;
}

// Fonts for Displaying Program Code in LATEX
// Adrian P. Robson, nepsweb.co.uk
// 8 October 2012
// http://nepsweb.co.uk/docs/progfonts.pdf

\end{lstlisting}

\subsection{Notasi}

Simbol-simbol (matematika) yang sering digunakan sepanjang penulisan skripsi, dapat dimasukkan ke dalam ``Daftar Notasi''. Daftar ini ada di halaman depan sebelum Bab~\ref{chap:intro}.
Cara memasukkan sebuah simbol ke dalam Daftar Notasi adalah menggunakan perintah \verb|\nomenclature|. Contoh:
\begin{center}
    \verb|\nomenclature[]{$A$}{luas kandang ular}|    
\end{center}
\nomenclature[]{$A$}{luas kandang ular}
\nomenclature[]{$n$}{banyaknya ular}
\nomenclature[]{$k$}{jumlah kepala per seekor ular\nomrefpage}
Argumen opsional digunakan untuk mengurutkan notasi. Silahkan lihat sendiri dokumentasi package \verb|nomencl|

