\chapter{Analisis}
\label{chap:analisis}
Bab ini membahas tentang analisis terkait evaluasi yang akan dilakukan pada SharIF Judge. 


\section{Metodologi Pemeriksaan}
\label{sec:metodologi}

Sebelum menentukan metodologi yang akan digunakan, ada beberapa hal yang perlu dipertimbangkan terlebih dahulu. Hal ini akan memengaruhi cara serta hasil akhir pemeriksaan dan evaluasi. Hal-hal tersebut antara lain sebegai berikut.
\begin{itemize}
	%\item Jumlah aturan dalam PSR berjumlah ... sehingga akan membutuhkan banyak waktu untuk memeriksa seluruh \textit{file} PHP dalam SharIF Judge. 
	\item Jumlah \textit{file} PHP dari SharIF Judge yang diperiksa berjumlah 34. Jumlah ini didasarkan pada jumlah \textit{file} yang memiliki tanda ``SharIF Judge'' di dalamnya. \textit{File}-\textit{file} yang dibuat oleh \textit{framework} atau \textit{library} lainnya tidak diperiksa. Walaupun ada kemungkinan bahwa telah dilakukan perubahan atau tambahan kode pada \textit{file}-\textit{file} yang dibuat \textit{framework}, pemeriksaan tetap tidak dilakukan karena akan cukup sulit dan memakan waktu yang sangat banyak untuk mencarinya satu per satu. 
	\item Saat dokumen ini dibuat, hanya tersedia \textit{tools} yang dapat membantu memeriksa penulisan kode sesuai PSR-12. Selain dari itu pemeriksaan harus dilakukan secara manual.
	\item Beberapa aturan PSR mungkin tidak relevan dengan \textit{file} PHP yang diperiksa. Sebagai contoh, tidak semua \textit{file} PHP menggunakan fungsi yang berkaitan dengan waktu dan tanggal dalam kodenya sehingga PSR-20: Clock tidak relevan dalam pemeriksan \textit{file}-\textit{file} tersebut.
\end{itemize} 
Dengan pertimbangan-pertimbangan tersebut, maka metodologi pemeriksaan yang dilakukan adalah sebagai berikut.
\begin{enumerate}
	\item Melakukan pemeriksaan secara otomatis sesuai rekomendasi PSR-12 dengan bantuan PHP CS Fixer.
	\item Melakukan pemeriksaan pemenuhan rekomendasi PSR lainnya secara manual tanpa PHP CS Fixer. 
\end{enumerate}


\section{Tingkat Kepatuhan SharIF Judge terhadap PSR}
\label{sec:tingkat_patuh}
Jumlah \textit{file} PHP yang diperiksa adalah 34, di luar \textit{file}-\textit{file} bawaan CodeIgniter. Setiap \textit{file} tersebut diperiksa sesuai kebutuhannya karena tidak semua rekomendasi PSR relevan terhadap kode program. Dalam prosesnya, pemeriksaan dibantu dengan PHP CS Fixer. 

\subsection{Pemeriksaan dengan PHP CS Fixer}
\label{subsec:periksa_otomatis}
PHP CS Fixer menyediakan fitur untuk menampilkan kode program yang diperiksa beserta dengan rekomendasi perubahannya sesuai dengan \textit{ruleset} yang sudah ditentukan sebelumnya, di mana \textit{ruleset} yang dipakai yaitu PSR-01 dan PSR-12. Maka dari itu, PHP CS Fixer sudah memeriksa dan memberikan rekomendasi perbaikannya sekaligus. Berikut adalah salah satu contoh ketika melakukan pemeriksaan pada \textit{Dashboard.php}.  Setelah menjalankan perintah dan \textit{flag} yang ditentukan sesuai contoh pada~\ref{subsubsec:phpcsfixer-flag}, keluaran yang dihasilkan dapat dilihat pada Kode~\ref{ck:31} dan~\ref{ck:32}. 

\begin{lstlisting}[language=diff, basicstyle=\ttfamily, frame=single,
	columns=fullflexible, showtabs=true, keepspaces=true, breaklines=true, label=ck:31, caption=Contoh penggunaan PHP CS Fixer pada sebagian kode \textit{Dashboard.php} dengan mengaktifkan parameter \textit{showtabs} yang menampilkan jarak \textit{tab}]
@@ -4,66 +4,70 @@
* @file Dashboard.php
* @author Mohammad Javad Naderi <mjnaderi@gmail.com>
*/
- defined('BASEPATH') OR exit('No direct script access allowed');
+ defined('BASEPATH') or exit('No direct script access allowed');

class Dashboard extends CI_Controller
{
+    public function __construct()
+    {
+        parent::__construct();
+        if (! $this->db->table_exists('sessions')) {
+            redirect('install');
+        }
+        if (! $this->session->userdata('logged_in')) { // if not logged in
+            redirect('login');
+        }
+        $this->load->model('notifications_model')->helper('text');
+    }
	
	
- public function __construct()
-	{
-		parent::__construct();
-		if ( ! $this->db->table_exists('sessions'))
-			redirect('install');
-		if ( ! $this->session->userdata('logged_in')) // if not logged in
-			redirect('login');
-		$this->load->model('notifications_model')->helper('text');
-	}

\end{lstlisting}

\begin{lstlisting}[language=diff, basicstyle=\ttfamily, frame=single,
	columns=fullflexible, showspaces=true, keepspaces=true, breaklines=true, label=ck:32, caption=Contoh penggunaan PHP CS Fixer pada sebagian kode \textit{Dashboard.php} dengan mengaktifkan parameter \textit{showspaces} yang menampilkan jarak \textit{space}]
@@ -4,66 +4,70 @@
* @file Dashboard.php
* @author Mohammad Javad Naderi <mjnaderi@gmail.com>
*/
- defined('BASEPATH') OR exit('No direct script access allowed');
+ defined('BASEPATH') or exit('No direct script access allowed');

class Dashboard extends CI_Controller
{
+    public function __construct()
+    {
+        parent::__construct();
+        if (! $this->db->table_exists('sessions')) {
+            redirect('install');
+        }
+        if (! $this->session->userdata('logged_in')) { // if not logged in
+            redirect('login');
+        }
+        $this->load->model('notifications_model')->helper('text');
+    }
	
	
- public function __construct()
-	{
-		parent::__construct();
-		if ( ! $this->db->table_exists('sessions'))
-			redirect('install');
-		if ( ! $this->session->userdata('logged_in')) // if not logged in
-			redirect('login');
-		$this->load->model('notifications_model')->helper('text');
-	}
		
\end{lstlisting}

Jika diperiksa dengan seksama, kode asli (bertanda ``-'') memiliki jarak indentasi satu kali \textit{tab} dari sisi kiri awal penulisan yang mana seharusnya dimulai dengan empat spasi untuk setiap level indentasi seperti yang ada dalam PSR-12: ``Kode HARUS menggunakan indentasi sebanyak 4 spasi untuk setiap level indentasi, dan TIDAK BOLEH menggunakan \textit{tab}-\textit{tab} untuk indentasi''. Hal ini mungkin tidak terlihat secara sekilas oleh mata sehingga harus diperiksa kembali setiap indentasinya. Hasil rekomendasi perbaikan yang diberikan keluaran (bertanda ``+'') di atas menampilkan indentasi sepanjang empat spasi dan kode aslinya (bertanda ``-'') sepanjang satu \textit{tab}. Hal ini akan lebih jelas jika diperiksa secara langsung melalui terminal. 
Sejauh ini, semua \textit{file} sudah diperiksa dengan PHP CS Fixer dan hasil keluarannya lebih lengkap ada di Lampiran dokumen Tugas Akhir karena terlalu panjang jika dicantumkan semuanya di Progress Report. Secara umum, standar yang belum terpenuhi sama dengan contoh di atas yaitu penggunaan \textit{tab} yang seharusnya menggunakan empat spasi. 			


\subsection{Pemeriksaan Tanpa PHP CS Fixer}
\label{periksa_manual}
Setelah pemeriksaan dengan bantuan PHP CS Fixer, selanjutnya dilakukan pemeriksaan sesuai rekomendasi PSR selain PSR-12 pada setiap \textit{file} PHP secara manual. Setiap hasil pemeriksaan akan ditandai sebagai ``Terpenuhi'' jika rekomendasi PSR sudah terpenuhi, ``Belum terpenuhi'' jika rekomendasi PSR belum terpenuhi, dan ``Tidak Relevan'' jika rekomendasi dari PSR tersebut tidak relevan dengan kode program yang diperiksa.

\subsubsection{PSR-01: Basic Coding Standard}
Selain menggunakan PHP CS Fixer, pemeriksaan PSR-01 juga dilakukan secara manual untuk memastikan kembali bahwa rekomendasinya benar terpenuhi.
\begin{enumerate}
	\item Assignments.php (Belum Terpenuhi)
	
	Ada beberapa metode yang tidak dideklarasikan dengan \texttt{camelCase}, yaitu \verb|downloadtestsdesc()|,  \verb|download_submissions()|, dan \verb|_add|.
	
	\item Dashboard.php (Belum Terpenuhi)
	
	Ada metode yang tidak dideklarasikan dengan \texttt{camelCase}, yaitu \verb|widget_positions()|.
	
	\item Halloffame.php (Belum Terpenuhi)
	
	Ada metode yang tidak dideklarasikan dengan \texttt{camelCase}, yaitu \verb|hof_details()|.
	
	\item Install.php (Terpenuhi)
	\item Login.php (Terpenuhi)
	
	Ada metode yang tidak dideklarasikan dengan \texttt{camelCase}, yaitu \verb|_registration_code()|.
	
	\item Logs.php (Terpenuhi)
	\item Moss.php (Belum Terpenuhi)
	
	Ada metode yang tidak dideklarasikan dengan \texttt{camelCase}, yaitu \verb|_detect()|.
	
	\item Notifications.php (Terpenuhi)
	\item Problems.php (Terpenuhi)
	\item Profile.php (Belum Terpenuhi)
	
	Ada metode yang tidak dideklarasikan dengan \texttt{camelCase}, yaitu \verb|_password_check()|.
	%ada yg lain
	
	\item Queue.php (Belum Terpenuhi)
	
	Ada metode yang tidak dideklarasikan dengan \texttt{camelCase}, yaitu \verb|empty_queue()|.
	
	\item Queueprocess.php (Terpenuhi)
	\item Rejudge.php (Belum Terpenuhi)
	
	Ada metode yang tidak dideklarasikan dengan \texttt{camelCase}, yaitu \verb|rejudge_single()|.
	
	\item Scoreboard.php (Terpenuhi)
	\item Server\_time.php (Terpenuhi)
	\item Settings.php (Terpenuhi)
	\item Submissions.php (Belum Terpenuhi)
	
	Ada metode yang tidak dideklarasikan dengan \texttt{camelCase}, yaitu \verb|_download_excel()|.
	%ada yg lain
	\item Submit.php (Belum Terpenuhi)
	
	Ada metode yang tidak dideklarasikan dengan \texttt{camelCase}, yaitu \verb|_language_to_type()|.
	%ada yg lain
	\item Users.php (Belum Terpenuhi)
	
	Ada metode yang tidak dideklarasikan dengan \texttt{camelCase}, yaitu \verb|delete_submissions()|.
	%ada yg lain
	\item shj\_helper.php (Terpenuhi)
	\item form\_validation\_lang.php (Terpenuhi)
	\item MY\_Form\_validation.php (Terpenuhi)
	\item MY\_Profiler.php (Terpenuhi)
	\item Shj\_pagination.php (Belum Terpenuhi)
	
	Ada metode yang tidak dideklarasikan dengan \texttt{camelCase}, yaitu \verb|create_links()|.
	
	\item Assignment\_model.php (Belum Terpenuhi)
	
	Ada metode yang tidak dideklarasikan dengan \texttt{camelCase}, yaitu \verb|add_assignment()|. %ada yg lain
	
	\item Hof\_model.php (Belum Terpenuhi)
	
	Ada metode yang tidak dideklarasikan dengan \texttt{camelCase}, yaitu \verb|get_all_final_submission()|. %ada yg lain
	
	\item Logs\_model.php (Belum Terpenuhi)
	
	Ada metode yang tidak dideklarasikan dengan \texttt{camelCase}, yaitu \verb|insert_to_logs()|. %ada yg lain
	
	\item Notifications\_model.php (Belum Terpenuhi)
	
	Ada metode yang tidak dideklarasikan dengan \texttt{camelCase}, yaitu \verb|get_all_notifications()|. %ada yg lain
	
	\item Queue\_model.php (Belum Terpenuhi)
	
	Ada metode yang tidak dideklarasikan dengan \texttt{camelCase}, yaitu \verb|in_queue()|. %ada yg lain
	
	\item Scoreboard\_model.php (Belum Terpenuhi)
	
	Ada metode yang tidak dideklarasikan dengan \texttt{camelCase}, yaitu \verb|_generate_scoreboard()|. %ada yg lain
	
	\item Settings\_model.php (Belum Terpenuhi)
	
	Ada metode yang tidak dideklarasikan dengan \texttt{camelCase}, yaitu \verb|get_setting()|. %ada yg lain
	
	\item Submit\_model.php (Belum Terpenuhi)
	
	Ada metode yang tidak dideklarasikan dengan \texttt{camelCase}, yaitu \verb|get_submission()|. %ada yg lain
	
	\item User\_model.php (Belum Terpenuhi)
	
	Ada metode yang tidak dideklarasikan dengan \texttt{camelCase}, yaitu \verb|have_user()|. %ada yg lain
	
	\item User.php (Belum Terpenuhi)
	
	Ada metode yang tidak dideklarasikan dengan \texttt{camelCase}, yaitu \verb|select_assignment()|. %ada yg lain
	
\end{enumerate}	


\subsubsection{PSR-03: Logger Interface}
Rekomendasi ini sudah ``Terpenuhi''. Implementasi \textit{Logger} ini tidak dapat diperiksa pada \textit{file} PHP satu per satu namun dapat dilihat pada \verb|application/vendor/psr/log/Psr/Log/|.



\subsubsection{PSR-04: Autoloading Standard}
Rekomendasi ini sudah ``Terpenuhi''. Implementasi \textit{Autoloading} ini tidak dapat diperiksa pada \textit{file} PHP satu per satu namun dapat dilihat dari \textit{composer.json} pada \verb|/container/|, \verb|/log/|, dan \verb|/simple-cache/| di dalam \verb|application/vendor/psr/|.


\subsubsection{PSR-06: Caching Interface}
Rekomendasi ini ``Tidak Relevan'' untuk diimplementasikan karena \textit{caching} keseluruhan ditangani oleh PSR-16: Simple Cache. 

\subsubsection{PSR-07: HTTP Message Interface}
Rekomendasi ini ``Belum Terpenuhi'' karena tidak diimplementasikannya \verb|Psr/Http/Message/RequestInterface| atau pun \verb|Psr/Http/Message/ResponseInterface|.


\subsubsection{PSR-11}
Rekomendasi ini sudah ``Terpenuhi''. Implementasi \textit{Container} ini tidak dapat diperiksa pada \textit{file} PHP satu per satu namun dapat dilihat pada \verb|application/vendor/psr/container/src/|.
%teliti lagi nanti

\subsubsection{PSR-13}
Rekomendasi ini ``Belum Terpenuhi'' karena tidak diimplementasikannya paket-paket dari \verb|Psr/Link/|.


\subsubsection{PSR-14}
Rekomendasi ini ``Belum Terpenuhi'' karena tidak diimplementasikannya \verb|Psr/EventDispacther/|.
%tapi ditangain adldap


\subsubsection{PSR-15: HTTP Server Request Handlers}
Rekomendasi ini ``Belum Terpenuhi'' karena tidak diimplementasikannya \verb|Psr/Http/Server/RequestHandlerInterface|.
%tapi ditangain pihak lain?

\subsubsection{PSR-16}
Rekomendasi ini sudah ``Terpenuhi''. Implementasi \textit{Cache} ini tidak dapat diperiksa pada \textit{file} PHP satu per satu namun dapat dilihat pada \verb|application/vendor/psr/simple-cache/src/|.


\subsubsection{PSR-17:HTTP Factories}
Rekomendasi ini ``Belum Terpenuhi'' karena tidak ada implementasi apa pun dari \verb|Psr/Http/Message/|. 

\subsubsection{PSR-18: HTTP Client}
Rekomendasi ini ``Belum Terpenuhi'' karena tidak ada implementasi apa pun dari \verb|ClientInterface|.

\subsubsection{PSR-20}
\begin{enumerate}
	\item Assignments.php (Tidak relevan) %cek lagi
	\item Dashboard.php (Tidak Relevan)
	\item Halloffame.php (Tidak Relevan)
	\item Install.php (Belum Terpenuhi)
	
	Ada fungsi waktu namun tidak menggunakan Clock Interface.
	
	\item Login.php (Tidak Relevan)
	\item Logs.php (Tidak Relevan)
	\item Moss.php (Belum Terpenuhi)
	
	Ada fungsi waktu namun tidak menggunakan Clock Interface.
	
	\item Notifications.php (Tidak Relevan)
	\item Problems.php (Tidak Relevan)
	\item Profile.php (Tidak Relevan)
	\item Queue.php (Tidak Relevan)
	\item Queueprocess.php (Tidak Relevan)
	\item Rejudge.php (Tidak Relevan)
	\item Scoreboard.php (Tidak Relevan)
	\item Server\_time.php (Tidak Relevan)
	\item Settings.php (Tidak Relevan)
	\item Submissions.php (Belum Terpenuhi)
	
	Ada fungsi waktu namun tidak menggunakan Clock Interface.
	
	\item Submit.php (Belum Terpenuhi)
	
	Ada fungsi waktu namun tidak menggunakan Clock Interface.
	
	\item Users.php (Belum Terpenuhi)
	
	Ada fungsi waktu namun tidak menggunakan Clock Interface.
	
	\item shj\_helper.php (Belum Terpenuhi)
	
	Ada fungsi waktu namun tidak menggunakan Clock Interface.
	
	\item form\_validation\_lang.php (Tidak Relevan)
	\item MY\_Form\_validation.php (Tidak Relevan)
	\item MY\_Profiler.php (Tidak Relevan)
	\item Shj\_pagination.php (Tidak Relevan)
	\item Assignment\_model.php (Tidak Relevan)
	\item Hof\_model.php (Tidak Relevan)
	\item Logs\_model.php (Tidak Relevan)
	\item Notifications\_model.php (Tidak Relevan)
	\item Queue\_model.php (Tidak Relevan)
	\item Scoreboard\_model.php (Tidak Relevan)
	\item Settings\_model.php (Tidak Relevan)
	\item Submit\_model.php (Tidak Relevan)
	\item User\_model.php (Belum Terpenuhi)
	
	Ada fungsi waktu namun tidak menggunakan Clock Interface.
	
	\item User.php (Tidak Relevan)
\end{enumerate}

