\chapter{Analisis}
\label{chap:analisis}
Bab ini membahas tentang analisis terkait evaluasi yang akan dilakukan pada SharIF Judge. 


\section{Metodologi Pemeriksaan}
\label{sec:metodologi}

Sebelum menentukan metodologi yang akan digunakan, ada beberapa hal yang perlu dipertimbangkan terlebih dahulu. Hal ini akan memengaruhi cara serta hasil akhir pemeriksaan dan evaluasi. Hal-hal tersebut antara lain sebegai berikut.
\begin{itemize}
	\item Jumlah aturan dalam PSR berjumlah ... sehingga akan membutuhkan banyak waktu untuk memeriksa seluruh \textit{file} PHP dalam SharIF Judge. Jumlah kode dalam \textit{file} PHP dari SharIF Judge yang perlu diperiksa berjumlah ... .
	\item Saat dokumen ini dibuat, hanya tersedia \textit{tools} yang dapat membantu memeriksa penulisan kode sesuai PSR-12. Selain dari itu pemeriksaan harus dilakukan secara manual.
	\item Beberapa aturan PSR mungkin tidak relevan dengan \textit{file} PHP yang diperiksa. Sebagai contoh, tidak semua \textit{file} PHP menggunakan fungsi waktu  dalam kodenya sehingga PSR-20: Clock tidak dibutuhkan dalam pemeriksan \textit{file}-\textit{file} tersebut.
\end{itemize} 

Dengan pertimbangan-pertimbangan tersebut, maka metodologi pemeriksaan yang dilakukan adalah sebagai berikut.
\begin{enumerate}
	\item Melakukan pemeriksaan secara otomatis sesuai rekomendasi PSR-12 dengan bantuan PHP CS Fixer.
	\item Melakukan pemeriksaan secara manual. 
\end{enumerate}


\section{Tingkat Kepatuhan SharIF Judge terhadap PSR}
\label{sec:tingkat_patuh}
Jumlah \textit{file} PHP yang diperiksa adalah 34, di luar \textit{file}-\textit{file} bawaan CodeIgniter. Setiap \textit{file} tersebut diperiksa sesuai kebutuhannya karena tidak semua rekomendasi PSR relevan terhadap kode program. Dalam prosesnya, pemeriksaan dibantu dengan PHP CS Fixer. 

\subsection{Pemeriksaan dengan PHP CS Fixer}
\label{subsec:periksa_otomatis}
PHP CS Fixer menyediakan fitur untuk menampilkan kode program yang diperiksa beserta dengan rekomendasi perubahannya sesuai dengan \textit{ruleset} yang sudah ditentukan sebelumnya yaitu PSR-12. Maka dari itu, PHP CS Fixer sudah memeriksa dan memberikan rekomendasi perbaikannya sekaligus. Setelah menjalankan perintah dan \textit{flag} yang ditentukan, keluaran yang dihasilkan dapat dilihat seperti contoh Kode~\ref{ck:31}.

\begin{lstlisting}[basicstyle=\ttfamily, frame=single,
	columns=fullflexible, keepspaces=true, breaklines=true, label=ck:31, caption=Contoh penggunaan PHP CS Fixer pada \textit{Dashboard.php}]
	
	Loaded config default.
	Using cache file ".php-cs-fixer.cache".
	1) application/controllers/Dashboard.php
	---------- begin diff ----------
	--- /Users/user/Documents/KULIAH/Skripsi/SharIF-Judge-Version-1/application/controllers/Dashboard.php
	+++ /Users/user/Documents/KULIAH/Skripsi/SharIF-Judge-Version-1/application/controllers/Dashboard.php
	@@ -4,66 +4,70 @@
	* @file Dashboard.php
	* @author Mohammad Javad Naderi <mjnaderi@gmail.com>
	*/
	-defined('BASEPATH') OR exit('No direct script access allowed');
	+defined('BASEPATH') or exit('No direct script access allowed');
	
	class Dashboard extends CI_Controller
	{
		+    public function __construct()
		+    {
			+        parent::__construct();
			+        if (! $this->db->table_exists('sessions')) {
				+            redirect('install');
				+        }
			+        if (! $this->session->userdata('logged_in')) { // if not logged in
				+            redirect('login');
				+        }
			+        $this->load->model('notifications_model')->helper('text');
			+    }
		
		
		-       public function __construct()
		-       {
			-               parent::__construct();
			-               if ( ! $this->db->table_exists('sessions'))
			-                       redirect('install');
			-               if ( ! $this->session->userdata('logged_in')) // if not logged in
			-                       redirect('login');
			-               $this->load->model('notifications_model')->helper('text');
			-       }
		+    // ------------------------------------------------------------------------
		
		
		-       // ------------------------------------------------------------------------
		+    public function index()
		+    {
			+        $data = array(
			+            'all_assignments' => $this->assignment_model->all_assignments(),
			+            'week_start' => $this->settings_model->get_setting('week_start'),
			+            'wp' => $this->user->get_widget_positions(),
			+            'notifications' => $this->notifications_model->get_latest_notifications()
			+        );
			
			+        // detecting errors:
			+        $data['errors'] = array();
			+        if($this->user->level === 3) {
				+            $path = $this->settings_model->get_setting('assignments_root');
				+            if (! file_exists($path)) {
					+                array_push($data['errors'], 'The path to folder "assignments" is not set correctly. Move this folder somewhere not publicly accessible, and set its full path in Settings.');
					+            } elseif (! is_writable($path)) {
					+                array_push($data['errors'], 'The folder <code>"'.$path.'"</code> is not writable by PHP. Make it writable. But make sure that this folder is only accessible by you. Codes will be saved in this folder!');
					+            }
				
				-       public function index()
				-       {
					-               $data = array(
					-                       'all_assignments'=>$this->assignment_model->all_assignments(),
					-                       'week_start'=>$this->settings_model->get_setting('week_start'),
					-                       'wp'=>$this->user->get_widget_positions(),
					-                       'notifications' => $this->notifications_model->get_latest_notifications()
					-               );
					+            $path = $this->settings_model->get_setting('tester_path');
					+            if (! file_exists($path)) {
						+                array_push($data['errors'], 'The path to folder "tester" is not set correctly. Move this folder somewhere not publicly accessible, and set its full path in Settings.');
						+            } elseif (! is_writable($path)) {
						+                array_push($data['errors'], 'The folder <code>"'.$path.'"</code> is not writable by PHP. Make it writable. But make sure that this folder is only accessible by you.');
						+            }
					+        }
				
				-               // detecting errors:
				-               $data['errors'] = array();
				-               if($this->user->level === 3){
					-                       $path = $this->settings_model->get_setting('assignments_root');
					-                       if ( ! file_exists($path))
					-                               array_push($data['errors'], 'The path to folder "assignments" is not set correctly. Move this folder somewhere not publicly accessible, and set its full path in Settings.');
					-                       elseif ( ! is_writable($path))
					-                               array_push($data['errors'], 'The folder <code>"'.$path.'"</code> is not writable by PHP. Make it writable. But make sure that this folder is only accessible by you. Codes will be saved in this folder!');
					+        $this->twig->display('pages/dashboard.twig', $data);
					+    }
				
				-                       $path = $this->settings_model->get_setting('tester_path');
				-                       if ( ! file_exists($path))
				-                               array_push($data['errors'], 'The path to folder "tester" is not set correctly. Move this folder somewhere not publicly accessible, and set its full path in Settings.');
				-                       elseif ( ! is_writable($path))
				-                               array_push($data['errors'], 'The folder <code>"'.$path.'"</code> is not writable by PHP. Make it writable. But make sure that this folder is only accessible by you.');
				-               }
			
			-               $this->twig->display('pages/dashboard.twig', $data);
			-       }
		+    // ------------------------------------------------------------------------
		
		+    /**
		+     * Used by ajax request, for saving the user's Dashboard widget positions
		+     */
		+    public function widget_positions()
		+    {
			+        if (! $this->input->is_ajax_request()) {
				+            show_404();
				+        }
			+        if ($this->input->post('positions') !== null) {
				+            $this->user->save_widget_positions($this->input->post('positions'));
				+        }
			+    }
		
		-       // ------------------------------------------------------------------------
		-
		-       /**
		-        * Used by ajax request, for saving the user's Dashboard widget positions
		-        */
		-       public function widget_positions()
		-       {
			-               if ( ! $this->input->is_ajax_request() )
			-                       show_404();
			-               if ($this->input->post('positions') !== NULL)
			-                       $this->user->save_widget_positions($this->input->post('positions'));
			-       }
		-
		-}
	\ No newline at end of file
	+}

----------- end diff -----------
\end{lstlisting}

\subsection{Pemeriksaan Manual}
\label{periksa_manual}