\chapter{Analisis}
\label{chap:analisis}
Bab ini membahas tentang analisis terkait evaluasi yang akan dilakukan pada SharIF Judge. 


\section{Metodologi Pemeriksaan}
\label{sec:metodologi}

Sebelum menentukan metodologi yang akan digunakan, ada beberapa hal yang perlu dipertimbangkan terlebih dahulu. Hal ini akan memengaruhi cara serta hasil akhir pemeriksaan dan evaluasi. Hal-hal tersebut antara lain sebegai berikut.
\begin{itemize}
	\item Jumlah aturan dalam PSR berjumlah ... sehingga akan membutuhkan banyak waktu untuk memeriksa seluruh \textit{file} PHP dalam SharIF Judge. Jumlah kode dalam \textit{file} PHP dari SharIF Judge yang perlu diperiksa berjumlah ... .
	\item Saat dokumen ini dibuat, hanya tersedia \textit{tools} yang dapat membantu memeriksa penulisan kode sesuai PSR-12. Selain dari itu pemeriksaan harus dilakukan secara manual.
	\item Beberapa aturan PSR mungkin tidak relevan dengan \textit{file} PHP yang diperiksa. Sebagai contoh, tidak semua \textit{file} PHP menggunakan fungsi waktu  dalam kodenya sehingga PSR-20: Clock tidak dibutuhkan dalam pemeriksan \textit{file}-\textit{file} tersebut.
\end{itemize} 

Dengan pertimbangan-pertimbangan tersebut, maka metodologi pemeriksaan yang dilakukan adalah sebagai berikut.
\begin{enumerate}
	\item 
\end{enumerate}

\section{Tingkat Kepatuhan SharIF Judge terhadap PSR}
\label{sec:tingkatpatuh}