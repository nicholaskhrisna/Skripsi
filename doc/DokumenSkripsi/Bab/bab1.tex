%versi 2 (8-10-2016) 
\chapter{Pendahuluan}
\label{chap:intro}
   
\section{Latar Belakang}
\label{sec:label}

SharIF Judge merupakan perangkat lunak berbasis web yang dapat digunakan oleh mahasiswa Teknik Informatika UNPAR. SharIF Judge dibuat oleh Mohammad Javad Naderi. Mahasiswa dapat mengunggah kode program dan akan dinilai kebenarannya secara otomatis oleh SharIF Judge.

PHP Standards Recommendations (PSR) adalah kumpulan standar penulisan PHP yang dibuat oleh PHP Framework Interop Group. Pada saat skripsi ini dibuat, terdapat 14 standar yang sudah diterima dan digunakan.

Pada skripsi ini, keseluruhan PHP pada SharIF Judge akan dilihat dan dievaluasi seberapa jauh standar PSR yang sudah dipenuhi. Alat yang digunakan adalah PHP linter yang dapat membantu proses pemeriksaan sesuai salah satu standar, yaitu Extended Coding Style Guide atau aturan penulisan PHP yang sudah diperbarui. Sementara itu standar lainnya diperiksa secara manual. 

\dtext{5-10}

\section{Rumusan Masalah}
\label{sec:rumusan}
Rumusan masalah yang akan dibahas pada skripsi ini sebagai berikut:
\begin{itemize}
	\item Apakah SharIF Judge sudah memenuhi PSR?
	\item Bagaimana mengevaluasi SharIF Judge agar sesuai dengan PSR?  
\end{itemize}

\dtext{6}

\section{Tujuan}
\label{sec:tujuan}
Tujuan yang ingin dicapai dalam penulisan skripsi ini sebagai berikut:
\begin{itemize}
	\item Mengetahui seberapa jauh PSR yang sudah terpenuhi pada SharIF Judge.
	\item Membuat evaluasi SharIF Judge agar sesuai dengan PSR.
\end{itemize}

\dtext{7}

\section{Batasan Masalah}
\label{sec:batasan}
Untuk mempermudah pembuatan template ini, tentu ada hal-hal yang harus dibatasi, misalnya saja bahwa template ini bukan berupa style \LaTeX{} pada umumnya (dengan alasannya karena belum mampu jika diminta membuat seperti itu)

\dtext{8}

\section{Metodologi}
\label{sec:metlit}
Metode penelitian yang akan digunakan dalam skripsi ini adalah:
\begin{enumerate}
	\item Memperlajari SharIF Judge saat ini
	\item Melakukan studi literatur mengenai PHP linter dan PSR
	\item Mengevaluasi PHP dari SharIF Judge sesuai dengan PSR
	\item Menguji SharIF Judge yang sudah dievaluasi
	\item Menulis dokumen skripsi
\end{enumerate}

\dtext{9}

\section{Sistematika Pembahasan}
\label{sec:sispem}
Untuk penulisan skripsi ini akan dibagi dalam lima bagian sebagai berikut:
Bab 1 Pendahuluan
Bab 2 Landasan Teori 
Bab 3 
Bab 4 
Bab 5 
\dtext{10}